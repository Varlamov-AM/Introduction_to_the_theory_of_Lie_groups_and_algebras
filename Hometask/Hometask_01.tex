\documentclass[10pt,a4paper]{article}
\usepackage[utf8]{inputenc}
\usepackage[russian]{babel}
\usepackage[OT1]{fontenc}
\usepackage{amsmath}
\usepackage{amsfonts}
\usepackage{amssymb}
\usepackage[dvipsnames]{xcolor}
\usepackage{graphicx}
\graphicspath{{Images/}}
\usepackage[left=2cm,right=2cm,top=2cm,bottom=2cm]{geometry}
\usepackage{calc}
\usepackage{wrapfig}
\usepackage{setspace}
\usepackage{indentfirst}
\usepackage{subfigure}
\usepackage{multirow}
\usepackage{amsfonts}
\usepackage{hyperref}
\hypersetup{
    pdfstartview=FitH,  
    linkcolor=black,
    urlcolor=red, 
    colorlinks=true,
    citecolor=blue}
\usepackage{tikz}
\usetikzlibrary{ decorations.markings}

\title{Введение в теорию групп и алгебр Ли \\
	\begin{normalsize}
		Домашнее задание №1
	\end{normalsize}}
\date{\today}

\author{Варламов Антоний Михайлович}

\begin{document}
	\maketitle
	
	\paragraph{Задача 1.} На лекции давалось такое определение группы:
	\begin{enumerate}
		\item Ассоциативность группового умножения
		\item Существование хотя бы одной правой единицы,
		$g\cdot e = g \ \forall \ g \in G$
		\item Существование $\ \forall \ g \in G$ хотя бы одного правого 
		элемента: $g\cdot g^{-1} = e$
	\end{enumerate}	
		
		Исходя из такого определения доказать:
		
	\begin{itemize}
		\item[a)] единственность единичного элемента $e$ и его перестановочность
		с любыми элементом группы.
		\item[б)] единственность обратного элемента $g^{-1}$ и его 
		перестановочность  с элементом $g$. 
	\end{itemize}
	
	\textit{Решение:}
	
	а)
	
	Сначала докажем единственность единицы. Предположим, что $\ \exists \ e_{1}, 
	e_{2}: \ \ \forall g \in G: ge_{1} = g, ge_{2} = g, e_{1} \neq e_{2}$. 
	Рассмотрим выражение 
	
	\begin{equation}
		e_{2}e_{1} \neq e_{2}e_{1}\label{eq:ex_1_eq_1}
	\end{equation}
	
	Действительно,$e_{2}e_{1} = e_{2}$, а $e_{1}e_{2} = e_{1}$, что с учетом 
	предположения обращает  (\ref{eq:ex_1_eq_1}) в верное неравенство. Домножим 
	(\ref{eq:ex_1_eq_1}) на какой-либо произвольный элемент группы $g$:

	\begin{equation}
		\label{eq:ex_1_eq_2}	
		g\cdot e_{2}e_{1} \neq g \cdot e_{1}e{2}
	\end{equation}		
	Рассмотрим выражение с учетом определения единичных элементов:
	
	\begin{equation}
		\label{eq:ex_1_eq_3}
		g e_{2}e_{1} \neq g e_{1}e_{2}
	\end{equation}
	\begin{equation}
		\label{eq:ex_1_eq_4}
		\left(g e_{2}\right)e_{1} \neq \left(g e_{1}\right) e_{2}
	\end{equation}
	\begin{equation}
		\label{eq:ex_1_eq_5}
		ge_{1} \neq ge_{2} 
	\end{equation}
	\begin{equation}
		\label{eq:ex_1_eq_6}
		g \neq g
	\end{equation}	
	
	Последнее неравенство очевидно неверно. Значит неравенства 	
	(\ref{eq:ex_1_eq_1}) - (\ref{eq:ex_1_eq_5}) также неверны. В таком случае
	рассмотрим еще раз неравенство (\ref{eq:ex_1_eq_1}):
	
	\begin{equation}
		\label{eq:ex_1_eq_7}
		e_{2}e_{1} = e_{1}e_{2} \Rightarrow e_{2} = e_{1}
	\end{equation}
	
	Пришли к противоречию с предположением, значит предположение неверно. 
	Заключаем, что \\$\ \exists ! e \in G: \ \forall \ g \in G: ge = g$
	
	Докажем, что единица коммутирует со всеми элементами. Для этого проведем 
	следующую цепочку рассуждений:
	
	\begin{equation}
		\label{eq:ex_1_eq_8}
		e = e\cdot e = e \left(gg^{-1}\right) = \left(eg\right)g^{-1} = e
	\end{equation}
	
	По определению обратного элемента (который пока предположим существует и 
	единственнен, а позже докажем) справедливо соотношение:
	
	\begin{equation}
		\label{eq:ex_1_eq_9}
		eg = e
	\end{equation}
	
	что означает коммутативность групповой единицы.
	
	\newpage
	б)
	
	Сначала докажем коммутативность обратного элемента. Для этого проведем 
	следующую цепочку рассуждений:
	
	\begin{equation}
		\label{eq:ex_1_eq_10}
		g = eg = \left(gg^{-1}\right)g = g\left(g^{-1}g\right) = g
	\end{equation}
	
	Согласно определению групповой единицы, получаем: 
	
	\begin{equation}
		\label{eq:ex_1_eq_11}
		g^{-1}g = e
	\end{equation}
	
	Что доказывает коммутативность обратного элемента. Докажем теперь его 
	единственность:
	
	Пусть $\ \forall \ g \in G: \exists \ g_{1}^{-1}, g_{2}^{-1} \in G: 
	gg_{1}= e, gg_{2} = e, g_{1}^{-1} \neq g_{2}^{-1}$
	
	Проведем следующую цепочку рассуждений:
	
	\begin{align}
		\label{eq:ex_1_eq_12}
		g_{1}^{-1} &\neq g_{2}^{-1}	\\
		\label{eq:ex_1_eq_13}
		\left(gg_{1}^{-1}\right)g_{1}^{-1} &\neq \left(gg_{1}^{-1}\right)
		g_{2}^{-1}	\\
		\label{eq:ex_1_eq_14}
		\left(g_{1}^{-1}g\right)g_{1}^{-1} &\neq \left(g_{1}^{-1}g\right)
		g_{2}^{-1}	\\	
		\label{eq:ex_1_eq_15}
		g_{1}^{-1} &\neq g_{1}^{-1}\left(gg_{2}^{-1}\right) \\
		\label{eq:ex_1_eq_16}
		g_{1}^{-1} &\neq g_{1}^{-1}
	\end{align}
	
	Пришли к противоречию, значит исходное предположение неверно. Значит, 
	обратный элемент единственнен и коммутирует для каждого элемента.
	
	\paragraph{Задача 2.} Рассмотрим целочисленные матрицы размера $2\times 2$
	$\begin{pmatrix}
		m & n \\
		p & q
	\end{pmatrix}$
	
	При каких условиях на числа $m, n, p, q$ эти матрицы образуют группу?\\
	
	\textit{Решение:}\\
	
	1. Определим, что является групповой единицей. Для любой матрицы должно 
	выполняться $eg = ge =g$. Таким свойствам удовлетворяет единичная матрица 
	$\mathbb{I}$. $\det \mathbb{I} = 1$.Вспомним свойство матриц, связанное с их
	 умножением:
	
	\begin{equation}
		\label{eq:ex_2_eq_01}
		\det\left(A\cdot B\right) = \det A \det B
	\end{equation}
	
	Так как матрицы целочисленные, а определитель содержит только операции 
	перемножения и сложения (без деления), то определитель целочисленной матрицы
	-- целое число. Значит
	\begin{equation}
		\label{eq:ex_2_eq_02}
		\det g \in \mathbb{Z}, \det g^{-1} \in \mathbb{Z} 
	\end{equation}
	
	Значит нужно решить в целых числах уравнение:
	
	\begin{equation}
		\label{eq:ex_2_eq_03}
		a\cdot b = 1
	\end{equation}
	
	В целых числах нас устроить только решение $a = b = 1$.
	
	Значит все целочисленные матрицы, образующие группу, имеют целый 
	определитель.
	
	Зададим все подобные матрицы параметрически:
	
	\begin{align*}
		\begin{pmatrix}
			1 & a \\
			0 & 1
		\end{pmatrix};
		\begin{pmatrix}
			1 & 0 \\
			b & 1
		\end{pmatrix};
		\begin{pmatrix}
			1 & a \\
			b & 1 + ab 
		\end{pmatrix};
		\begin{pmatrix}
			1 + ab & a \\
			b & 1
		\end{pmatrix}
	\end{align*}
	
	Нетрудно проверить, что матрицы, описанные выше образуют множество, 
	замкнутое относительное матричного умножения ($\ \forall \ a, b, \in 
	\mathbb{Z}$), Единичная матрица включена в это множество, а для любой 
	матрицы можно построить обратную. 
	
	\paragraph{Задача 3.} Доказать, что если любой элемент в группе $G$ 
	удовлетворяет свойству $g^{2} = e$, то $G$ -- абелева группа.\\
	
	\textit{Решение:}\\
	
	Рассмотрим два произвольных элемента $g, h \in G$. Предположим $gh \neq hg$.
	
	Проведем следующую цепочку рассуждений:
	
	\begin{align}
		\label{eq:ex_3_eq_01}
		gh &\neq hg\\
		\label{eq:ex_3_eq_02}
		gh\left(gh\right) &\neq hg\left(gh\right)\\
		\label{eq:ex_3_eq_03}
		\left(gh\right)\left(gh\right) &\neq h\left(gg\right)h\\
		\label{eq:ex_3_eq_04}
		\left(gh\right)\left(gh\right) &\neq hh\\
		\label{eq:ex_3_eq_05}
		\left(gh\right)\left(gh\right) &\neq e
	\end{align}
	
	Но $\left(gh\right) = f \in G$, Значит получаем неравенство $e \neq e$,
	которое очевидно неверно. Значит предположение ложно и любые два элемента 
	коммутируют, что и означает что группа является абелевой.
	
	\paragraph{Задача 5.} Пусть $S$ -- некоторое подмножество группы $G$.
	Подмножество $C\left(S\right) = \lbrace c | cs = sc , c \in G, s \in S
	\rbrace$ Называется \textit{централизатором} $S$. Подмножество $N\left(S
	\right) = \lbrace n | nSn^{-1}, n \in G\rbrace$ называется
	\textit{нормализатором} $S$.
	\begin{enumerate}
		\item Показать что централизатор является подгруппой $G$.
		\item Показать что нормализатор является подгруппой $G$ и 
		$C\left(S\right) \subset N\left(S\right)$.
		\item Показать, что если $S$ является подгруппой $G$, то $S \subset
		N\left(S\right)$ -- инвариантная подгруппа $N\left(S\right)$
	\end{enumerate}
	
	\textit{Решение:}\\
	
	1. Достаточно показать только замкнутость $C\left(S\right)$ (так как уже 
	доказано, что единичный и обратные коммутируют). Рассмотрим $c_{1}, c_{2} 
	\in C$
	
	\begin{align}
		\label{eq:ex_5_eq_01}
		c_{1}c_{2}s = sc_{1}c_{2}\\
		\label{eq:ex_5_eq_02}
		c_{1}sc_{2} = c_{1}sc_{2}
	\end{align}
	
	Что говорит и доказывает замкнутость. Значит централизатор $C\left(S\right)$
	-- подгруппа $G$.
	
	2. Покажем замкнутость нормализатора:
	
	\begin{equation}
		\label{eq:ex_5_eq_03}
		n_{1}n_{2}S\left(n_{1}n_{2}\right)^{-1} = 
		n_{1}n_{2}Sn_{2}^{-1}n_{1}^{-1} = 
		n_{1}\left(n_{2}Sn_{2}^{-1}\right)n_{1}^{-1} = 
		n_{1}Sn^{-1}_{1} = S
	\end{equation}
	
	Значит нормализатор замкнут. Единичный групповой элемент очевидно 
	принадлежит нормализатору, по определению своего действия на элементы 
	группы. Аналогично и обратные элементы, принадлежат нормализатору. 
	Докажем вложение $C\left(S\right) \subset N\left(S\right)$.
	
	Для этого проведем следующую цепочку рассуждений:
	
	\begin{equation}
		\label{eq:ex_5_eq_04}
		cSc^{-1} = cc^{-1}S = eS = S
	\end{equation}
	
	Что доказывает вложение.
	
	3. По определению нормальности подгруппы:
	
	\begin{equation}
		\label{eq:ex_5_eq_05}
		\forall \ n \in N\left(S\right): nSn^{-1} = S
	\end{equation}
	
	По определению нормализатора подмножества S, условие (\ref{eq:ex_5_eq_05})
	выполняется для всех $n \in N\left(S\right)$. Осталось доказать тот факт, 
	что если $S$ -- подгруппа $G$, она также будет являться подгруппой 
	$N\left(S\right)$. Докажем вложение:
	
	\begin{equation}
		\label{eq:ex_5_eq_06}
		\forall \ k \in S: kSk^{-1} = S
	\end{equation}
	
	свойство (\ref{eq:ex_5_eq_06}) выполняется в силу того, что $S$ -- 
	подгруппа (так как $ks \in S,$ если $k \in S, s\in S$). Значит вложение 
	выполнено, а инвариантность $S$ относительно $N\left(S\right)$ следует из 
	определения нормализатора.
	
	\paragraph{Задача 6.} Будем считать два элемента $g_{1}$ и $g_{2}$ из группы
	$G$ эквивалентными, если найдется такой элемент $g \in G: g_{1} = gg_{2}
	g^{-1}$. Доказать, что это действительно отношение эквивалентности.
	
	\textit{Решение:}\\
	
	Докажем рефлексивность:
	
	\begin{equation}
		\label{eq:ex_6_eq_01}
		g_{1} \sim g_{1}, \ \text{т.к.} g_{1} = eg_{1}e^{-1} = eg_{1}e
	\end{equation}
	
	Докажем симметричность:
	
	\begin{equation}
		\label{eq:ex_6_eq_02}
		g_{1} \sim g_{2} \Leftrightarrow g_{2}\sim g_{1}: g_{1} = gg_{2}g^{-2} 
		\Rightarrow g_{1}g = gg_{2}g^{-1}g \Rightarrow g^{-1}g_{1}g = g_{2}
	\end{equation}
	
	Докажем транзитивность:
	
	\begin{equation}
		\label{eq:ex_6_eq_03}
		g_{1} \sim g_{2}, g_{2} \sim g_{3} \Rightarrow g_{1} \sim g_{3}:
		g_{1} = g'g_{2}g'^{-1}, g_{2} = g''g_{3}g''^{-1} \Rightarrow 
		g_{1} = g'g''g_{3}g''^{-1}g'^{-1} = \left(g'g''\right)g_{3}
		\left(g'g''\right)^{-1}
	\end{equation}
	
	Значит, данное отношение действительно является отношением эквивалентности.
	
	\paragraph{Задача 7.}Пусть $H$ -- подгруппа группы $G$ и фактор-пространство 
	$G/H$ состоит из двух "точек" (смежных классов). Показать что $H$ -- 
	нормальная подгруппа.
	
	\textit{Решение:}\\
	
	\paragraph{Задача 8.} Доказать, что порядок подгруппы конечной группы делит
	порядок группы(Теорема Лагранжа)
	
	\textit{Решение:}\\
	
	Рассмотрим произвольный смежный класс по подгруппе $H$ группы $G$. Для 
	смежного класса справедливо утверждение, что они все равномощны подгруппе, 
	на которой эти смежные классы построены. Так как группа конечна, то и 
	подгруппа конечна (порядок подгруппы, как и порядок группы конечен), а также
	количество смежных классов конечно (так как количество элементов, из которых 
	можно построить смежные классы -- конечно). Значит, справедливо равенство:
	
	\begin{equation}
		\label{eq:ex_8_eq_01}
		\left|G\right| = k\left|H\right|
	\end{equation}
	
	где $k$ -- Количество смежных классов. Значит утверждение, приведенное в 
	условии справедливо.
	
	\paragraph{Задача 9.} Показать, что для $\ \forall \ g\in G$ конечной 
	группы выполняется: $g^{N} = e$
	
	\textit{Решение:}\\
	
	Вспомним, что порядок конечной группы определяет количество элементов (
	мощность носителя) группы. В таком случае, можно утверждать, что множество 
	содержит групповую единицу $e$ и еще  $N - 1$ элемент. Возьмем из этих 
	элементов произвольный и рассмотрим, что будет при возведении этого элемента
	в степень. Предположим, что выбранный элемент не удовлетворяет условию:
	$g^{N} \neq e$. Опишем, что это означает. будем считать, что на каждом 
	этапе возвести элемент в степень будем получать новый элемент (иначе 
	справедливо соотношение $g^{k - 1}g = g^{k - 1} \Rightarrow g^{k - 1} = e$)
	В таком случае мы имеем возможность получить не более $N - 1$ индивидуальных
	элемента. Дальнейшее умножение приведет к тому, что придется когда-то 
	получить обратный к элементу группы (в силу определения группы), что 
	приведет к появлению в последовательности групповой единицы, после чего 
	получение в последовательности новых элементов принципиально невозможно. 
	значит, что к $N - 1$ шагу для всех начальных элементов группы мы получим 
	обратный к предпоследнему или единицу, что говорит о том, что при следующей
	итерации обязательно выполнится $g^{N} = e$
	
	\paragraph{Задача 10.} Показать, что конечная группа простого порядка 
	является циклической группой.
	
	\textit{Решение:}\\
	
	Согласно выводам предыдущего задания, получаем, что конечная группа простого 
	порядка имеет только тривиальную подгруппу, и саму себя как подгруппу. В 
	таком случае $\ \forall \ g \in G, g \neq e: \ g^{k} \neq e, k < N$
	
	Иначе, можно было бы построить подгруппу $H = \lbrace h | h \in G, h
	= e, g, g^{2}, g^{3}, \ldots g^{k}, g^{-1}, g_{-2}, g^{-3}, \ldots
	g^{-k}\rbrace$, а значит порядок группы делился бы на $k$, чего не может 
	быть. Значит, в такой группе все элементы являются порождающими (кроме 
	групповой единицы) (следует напрямую из утверждения о невозможности 
	построить подгруппу, описанную выше).
	
	\paragraph{Задача 11.} Показать, что любая конечная группа с порядком не 
	более 5 является абелевой.
	
	\textit{Решение:}\\
	
	Тривиальные случаи с $G = \lbrace e \rbrace, \lbrace e, g, g^{-1}\rbrace$ 
	Рассматривать не будем, в силу очевидности справедливости утверждения.
	
	Рассмотрим группу 
	\begin{equation}
		\label{eq:ex_11_eq_01}
		G = \lbrace e, g_{1}, g_{2}, g^{-1}_{1}, g^{-1}_{2}\rbrace
	\end{equation}
	
	Коммутативность групповой единицы, а также соответствующих пар элемента и 
	его обратного очевидна (доказано в первом упражнении). Рассмотрим 
	"перекрестные" элементы: $g_{1}g_{2}^{-1}, g_{2}g_{1}^{-1}, g_{1}g_{2}, 
	g_{1}^{-1}g_{2}^{-1}$
	
	Сначала распишем $g_{1}g_{2}$:
	
	\begin{align}
		\label{eq:ex_11_eq_02}
		e &= \left(\left(g_{1}^{-1}g_{1}\right)\left(g_{2}g^{-1}_{2}\right)
		\right) = \left(\left(g_{2}g^{-1}_{2}\right)\left(g_{1}g_{1}^{-1}\right)
		\right)\\
		\label{eq:ex_11_eq_03}
		e &= g_{1}^{-1}\left(g_{1}g_{2}g_{2}^{-1}\right) = \left(g_{2}^{-1}g_{2}
		g_{1}\right)g_{1}^{-1}
	\end{align}
	
	Из коммутативности обратного элемента и уравнения (\ref{eq:ex_11_eq_03}) 
	немедленно следует, что $g_{1}g_{2} = g_{2}g_{1}$. 
	
	Выбирая в (\ref{eq:ex_11_eq_02}) подходящий порядок сомножителей, 
	нетрудно показать и остальные коммутативные равенства (для $g_{2}^{-1}g_{1}=
	g_{1}g_{2}^{-1}$ например вообще нет необходимости менять порядок (только 
	расставить в нужных местах скобки.))
	
	\paragraph{Задача 12.} Доказать, что у бесконечной группы число подгрупп 
	бесконечно.
	
	\textit{Решение:}
	
	Решение данной задачи в целом основывается на довольно простой идее: просто
	покажем, как можно построить не менее чем счетное число подгрупп. Для этого 
	выберем в группе счетный набор элементов и обратных к ним. Пронумеруем 
	элементы. Подгруппы будем строить по следующей схеме:
	
	\begin{equation}
		\label{eq:ex_12_eq_01}
		G_{g_{i}} = \lbrace e, g_{i}, g^{-1}_{i}\rbrace
	\end{equation}
	
	По построению таких группы не менее чем счетное количество, что и говорит 
	о том, что количество подгрупп бесконечно. (но только в случае наличия 
	возможности выбрать счетное множество элементов группы и обратные к ним).
	
	\paragraph{Задача 13.} Пусть $\mathbb{R}^{+}$ -- группа действительных чисел 
	по сложению, а $\mathbb{Z}$ -- подгруппа целых чисел. Описать фактор-
	пространство $\mathbb{R}^{+}/\mathbb{Z}$.
	
	\textit{Решение:}\\
	
	Рассмотрим отношение эквивалентности, порождаемое подгруппой целых чисел:
	
	\begin{equation}
		\label{eq:ex_13_eq_01}
		a \sim b \Leftrightarrow \ \exists \ z \in \mathbb{Z}: a = b + z
	\end{equation}
	
	Соответственно классом эквивалентности будет являться действительное число 
	в полуинтервале от 0 до 1.
	
	\begin{equation}
		\label{eq:ex_13_eq_02}
		\mathbb{R}^{+}/\mathbb{Z} = \left(0, 1\right]
	\end{equation}
	
	\paragraph{Задача 14.} Пусть $\mathbb{R}^{*}$ -- множество ненулевых 
	действительных чисел, рассматриваемое как группа по умножению, а 
	$\mathbb{R}^{+}$ -- подгруппа положительных действительных чисел. Найти 
	фактор-группу $\mathbb{R}^{*}/\mathbb{R}^{+}$
	
	\textit{Решение:}\\
	
	Аналогично предыдущей задаче:
	
	\begin{equation}
		\label{eq:ex_14_eq_01}
		a\sim b \Leftrightarrow \exists c: a = c\cdot b
	\end{equation}
	
	В силу того, что мы точно знаем группу и подгруппу, по которой будем 
	производить факторизацию, то в таком случае довольно легко найти фактор-
	группу:
	
	\begin{equation}
		\label{eq:ex_14_eq_02}
		\mathbb{R}^{*}/\mathbb{R}^{+} = \lbrace -1, 1\rbrace
	\end{equation}
	
	Как видно, фактор-группа действительно является группой (выполняются все 
	групповые аксиомы).
	
	\paragraph{Задача 15.} Пусть $\mathbb{C}^{*}$ -- множество ненулевых 
	комплексных чисел, рассматриваемое как группа по умножению, а $U
	\left(1\right)$ -- подгруппа комплексных чисел с единичным модулем. Найти
	фактор-группу $\mathbb{C}^{*}/U\left(1\right)$
	
	\textit{Решение:}\\
	
	Отношение эквивалентности, возникающее при подобной факторизации:
	
	\begin{equation}
		\label{eq:ex_15_eq_01}
		c_{1} \sim c_{2} \Leftrightarrow \ \exists c: c_{1} = cc_{2}
	\end{equation}
	
	Однако вместо комплексного числа можно взять действительное, а умножать 
	комплексные числа с одинаковым аргументом.
	
	В итоге:
 		
 	\begin{equation}
 		\label{eq:ex_15_eq_02}
 		\mathbb{C}^{*}/U\left(1\right) = \mathbb{R}^{+}/\lbrace 0 \rbrace
 	\end{equation}
 	
 	\paragraph{Задача 16.} Рассматривая множество целых чисел $\mathbb{Z}$ как
 	абелеву группу по сложению. Пусть $n$ -- фиксированное натуральное число, 
 	а $n\mathbb{Z} = \lbrace np | p \in Z\rbrace$ -- подгруппа $\mathbb{Z}$, 
 	состоящая из чисел, которые делятся без остатка.
 	
 	Показать, что фактор-группа $\mathbb{Z}/ n\mathbb{Z}$ -- циклическая группа 
 	порядка $n$
 	
 	\textit{Решение:}\\
 	
 	Покажем, что $1$ -- групповая единица фактор-группы, является порождающим 
 	элементом фактор-группы, т.е. $\ \forall n \in \mathbb{Z}/ n\mathbb{Z}: 
 	n = e^{n}$, где $e^{n} = e + e + e +\ldots + e = n \cdot e$.
 	
 	Действительно, по свойству групповой операции (сложение) описанное выше 
 	равенство становится верным. Такая фактор-группа называется группа вычетов 
 	по модулю $n$. Условие отсутствия нетривиальных подгрупп обеспечивается 
 	свойствами групповой операции (сложение).
 	
 	\paragraph{Задача 17.} Пусть $G$ --  множество действительных матриц вида:
 	$\begin{pmatrix}
 		d_{1} & a\\
 		0 & d_{2}
 	\end{pmatrix}$($a$ -- любое число, $d_{1}d_{2} \neq 0$
 	
 	\begin{enumerate}
 		\item Показать, что это множество образует матричную группу.
  		\item Найти центральную подгруппу. Найти фактор-группу группы $G$ по 
  		ее центральной подгруппе.
 		\item Показать, что множество $H$ матриц $\begin{pmatrix}
 			1 & a\\
 			0 & 1
 		\end{pmatrix}$ является инвариантной абелевой подгруппой группы $G$.
 		Найти фактор-группу $G/H$
 		\item Найти фактор-группу группы $G$по подгруппе, состоящих из матриц с 	
 		единичным детерминантом.
 	\end{enumerate}
 	
 	\textit{Решение:}\\
 	
 	Сначала докажем, что данное множество описывает группу.
 	
 	\begin{equation}
 		\label{eq:ex_17_eq_01}
 		\begin{pmatrix}
 			a_{1} & b\\
 			0 & a_{2}
 		\end{pmatrix}\cdot
 		\begin{pmatrix}
 			d_{1} & c\\
 			0 & d_{2}
 		\end{pmatrix} = 
 		\begin{pmatrix}
 			a_{1}d_{1} & a_{1}c + bd_{2}\\
 			0 & a_{2}d_{2}
 		\end{pmatrix}
 	\end{equation}
 	
 	в таком случае единичный элемент соответствует единичной матрице, а 
 	обратный для произвольной матрицы:
 	
 	\begin{equation}
 		g = 
 		\begin{pmatrix}
 			d_{1} & c\\
 			0 & d_{2}
 		\end{pmatrix}; \Rightarrow 
 		g^{-1} = 
 		\begin{pmatrix}
 			\frac{1}{d_{1}} & bd_{2} - a_{1}\\
 			0 & \frac{1}{d_{2}}
 		\end{pmatrix}
 	\end{equation}
 	
 	Для нахождения центральной подгруппы необходимо найти коммутирующую 
 	подгруппу. В данном случае таковой является группа матриц, которые 
 	являются диагональными. Общий вид фактор-группы: 
 	\begin{equation}
 		\label{eq:mil}
 		\begin{pmatrix}
 			1 & 0\\
 			0 & \frac{d_{2}}{d_{1}}
		\end{pmatrix}	 			
 	\end{equation} -- Диагональные матрицы.
 	
 	Рассмотрим множество матриц вида $\begin{pmatrix}
 		1 & a\\
 		0 & 1
 	\end{pmatrix}$. Докажем что это абелева группа:
 	
 	\begin{equation}
 		\label{eq:ex_17_eq_01}
		\begin{pmatrix}
 		1 & a\\
 		0 & 1
 	\end{pmatrix} \cdot 
		\begin{pmatrix}
 		1 & b\\
 		0 & 1
 	\end{pmatrix}  = 
		\begin{pmatrix}
 		1 & a + b\\
 		0 & 1
 	\end{pmatrix} 	
 	\end{equation}
 	
 	Как видно, результат не зависит от порядка сомножителей, значит подгруппа коммутативна. Так же она является инвариантной, т.к.
 	
 	\begin{equation}
 	\label{eq:ex_17_02}
 	\begin{pmatrix}
 		d_{1} & c\\
 		0 & d_{2}
 	\end{pmatrix}
 	\begin{pmatrix}
 		1 & a \\
 		0 & 1
 	\end{pmatrix}
 	\begin{pmatrix}
 		\frac{1}{d_{1}} & -\frac{a}{d_{1}}\\
 		0 & \frac{1}{d_{2}}
 	\end{pmatrix}  =   
 	\begin{pmatrix}
 		1 & -a + 1\\
 		0 & 1
 	\end{pmatrix}
	\end{equation} 	 
	
	Что говорит о том, что $H$ -- инвариантная абелева подгруппа.
	
	Фактор-группа -- группа матриц вида: $\begin{pmatrix}
		1 & c\\ 
		0 & \frac{d_{2}}{d_{1}}
	\end{pmatrix}$
	
	Фактор-группа группы $G$ по группам матриц с единичным детерминантом -- 
	группа матриц вида:$\begin{pmatrix}
		\frac{d_{1}}{d_{2}} & \frac{d_{1}d_{2}}{a}\\
		0 & \frac{d_{2}}{d_{1}}
	\end{pmatrix}$ 	
	
	\paragraph{Задача 18.} Является ли групповое множество однородным 
	пространством относительно левых (правых) сдвигов? Является ли групповое 
	множество однородным пространством относительно преобразований подобия?
	
	\textit{Решение:}\\
	
	Групповое множество является однородным	пространством относительно левых 
	(правых) сдвигов, так как сдвиги являются автоморфизмами. 
	
	\paragraph{Задача 19.} Рассмотрим $M^{3}$ -- трехмерное пространство 
	Минковского со скалярным произведением $\left(x, y\right) = x^{0}y^{0} - 
	x^{1}y^{1} - x^{2}y^{2}, x, y, \in M^{3}$. Множество всех обратимых 
	преобразований $M^{3}$, которые не изменяют указанное скалярное произведение
	называется трехмерной группой Лоренца. Описать все орбиты группы Лоренца в 
	пространстве $M^{3}$.
	
	\textit{Решение:}\\
	
	Для начала поймем, какие преобразования сохраняют метрику (скалярное 	
	произведение). Во-первых, это повороты плоскости $\mathbb{R^{2}}$, а также 
	так называемые Лоренцевы бусты. Все эти преобразования являются обратимыми.
	
	Форма орбит может быть описана с помощью геометрической интерпретации: 
	каждая орбита является пересечением нулевой плоскости, с гиперболоидом, 
	который может выродиться в так называемый световой конус. Орбиты в таком 
	случае становятся параболами, лежащими в соответствующих нулевых плоскостях.
 	
\end{document}