\documentclass[10pt,a4paper]{article}
\usepackage[utf8]{inputenc}
\usepackage[russian]{babel}
\usepackage[OT1]{fontenc}
\usepackage{amsmath}
\usepackage{amsfonts}
\usepackage{amssymb}
\usepackage[dvipsnames]{xcolor}
\usepackage{graphicx}
\graphicspath{{Images/}}
\usepackage[left=2cm,right=2cm,top=2cm,bottom=2cm]{geometry}
\usepackage{calc}
\usepackage{wrapfig}
\usepackage{setspace}
\usepackage{indentfirst}
\usepackage{subfigure}
\usepackage{multirow}
\usepackage{amsfonts}
\usepackage{hyperref}
\hypersetup{
    pdfstartview=FitH,  
    linkcolor=black,
    urlcolor=red, 
    colorlinks=true,
    citecolor=blue}
\usepackage{tikz}
\usetikzlibrary{ decorations.markings}

\title{Семинар 2}
\date{\today}

\author{Варламов Антоний Михайлович}
\newtheorem{thm}{Theorem}
\newtheorem{defin}{Definition}
\newtheorem{cons}{Consequence}
\newtheorem{exmpl}{Example}
\begin{document}
	\maketitle
	\tableofcontents
	
	\section{Гомоморфизмы}
	
	\begin{equation}	
		\Phi: G \rightarrow G'
	\end{equation}
	
	\begin{equation}
		\ker \Phi = \lbrace e\rbrace, \Im \Phi  = G'
	\end{equation}
	
	\begin{exmpl}
		Группа левых сдвигов: $G: g\rightarrow Lg$, $Lg\left(h\right) = gh$. 
		Группа левых сдвигов изморфна самой себе.
	\end{exmpl}
	
	Классификация гомоморфизмов строится на основе комбинации ядра и образа 
	гомоморфизма
	
	\begin{equation}
		\begin{matrix}
			Ker && Im \\
			G && \lbrace e' \rbrace \\
			<G && <G'\\
			\lbrace e \rbrace && G'
		\end{matrix}
	\end{equation}
	
	Пример накрытия: $G \supset H $ -- инвариантная, $\pi: G\rightarrow G/H$
	
	\begin{equation}
		g \rightarrow \hat{g}: G\rightarrow G
	\end{equation}
	
	\begin{equation}
		\hat{g}\left(h\right) = ghg^{-1}
	\end{equation}
	
	\begin{equation}
		g_{1}g_{2} \rightarrow \hat{g_{1}g_{2}}\left(h\right) = g_{1}g_{2}hg_{2}
		^{-1}g_{1}^{-1} = \hat{g_{1}}\left(\hat{g_{2}}\left(h\right)\right)
	\end{equation}
	
	В таком случае $Ker = C_{g}$ -- центр группы $G$
	
	\begin{equation}
		z = x + iy, 1 \rightarrow \begin{pmatrix}
		1 && 0 \\
		0 && 1
		\end{pmatrix}, i \rightarrow \begin{pmatrix}
		0 && -1\\
		1 && 0
		\end{pmatrix}
	\end{equation}
	
	\begin{thm}
		\begin{equation}
			\Phi: G\rightarrow G', Im \Phi \approx G/Ker \Phi
		\end{equation}
		
		\textit{Доказательство:} 
		$\psi : G/Ker \Phi \rightarrow G'$, в таком случае:
		
		\begin{equation}
			\psi\left(gK\right) = \Phi\left(g\right)
		\end{equation}
		\begin{equation}
			\psi\left(g_{1}Kg_{2}K\right) = \psi\left(g_{1}g_{2}K\right) = 
			\Phi\left(g_{1}g_{2}\right) = \Phi\left(g_{1}\right)
			\Phi\left(g_{2}\right) = 
			\psi\left(g_{1}K\right)\psi\left(g_{2}K\right)
		\end{equation}
		
		\begin{equation}
			\psi\left(K\right) = \Phi\left(e\right) = e'
		\end{equation}
	\end{thm}
	
	Пример применения:
	
	\begin{equation}
		G \rightarrow \hat{G}
	\end{equation}
	
	\begin{equation}
		\hat{G} \approx G/C_{g}
	\end{equation}
	
	Изоморфное отображение группы самой в себя называется автоморфизмом
	
	Группа всех автоморфизмов -- $AutG$
	
	Пример автоморфизма -- $\hat{G}$
	
	\begin{equation}
		AutG \supset \hat{G}
	\end{equation}
	
	Такая группа называется группой внутренних автоморфизмов
	
	Пусть $G \supset H$ -- нормальная подгруппа Тогда сопряжение элементами $H$
	-- внутренний автоморфизм, внешними будут сопряжения с элементами $G$
	
	Группа внутренних автоморфизмов --инвариантная подгруппа.
	
	Нужно показать $\Phi^{-1}\circ \hat{g}\circ \Phi$ -- преобразование подобия.
	
	\begin{eqnarray}
		\Phi\circ \hat{g}\circ \Phi^{-1}\ \left(h\right) = 
		\Phi\circ \hat{g} \circ \Phi^{-1}\left(\Phi\left(\widetilde{h}
		\right)\right) = \Phi\left(g\right)
	\end{eqnarray}
	
	\section{Алгебраическая классификация групп}
	
	Все группы делятся на два больших класса: полупростые и неполупростые
\end{document}