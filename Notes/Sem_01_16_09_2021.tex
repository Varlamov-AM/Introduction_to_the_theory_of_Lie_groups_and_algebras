\documentclass[10pt,a4paper]{article}
\usepackage[utf8]{inputenc}
\usepackage[russian]{babel}
\usepackage[OT1]{fontenc}
\usepackage{amsmath}
\usepackage{amsfonts}
\usepackage{amssymb}
\usepackage[dvipsnames]{xcolor}
\usepackage{graphicx}
\graphicspath{{Images/}}
\usepackage[left=2cm,right=2cm,top=2cm,bottom=2cm]{geometry}
\usepackage{calc}
\usepackage{wrapfig}
\usepackage{setspace}
\usepackage{indentfirst}
\usepackage{subfigure}
\usepackage{multirow}
\usepackage{amsfonts}
\usepackage{hyperref}
\hypersetup{
    pdfstartview=FitH,  
    linkcolor=black,
    urlcolor=red, 
    colorlinks=true,
    citecolor=blue}
\usepackage{tikz}
\usetikzlibrary{ decorations.markings}

\title{Семинар 1}
\date{\today}

\author{Варламов Антоний Михайлович}
\newtheorem{thm}{Theorem}
\newtheorem{defin}{Definition}
\newtheorem{cons}{Consequence}
\newtheorem{exmpl}{Example}
\begin{document}
	\maketitle
	\tableofcontents
	
	\section{Элементы теории абстрактных групп}
	
		\subsection{Определение абстрактной группы}
		
		Начнем изложение с приведения определения группы
		
		\begin{defin}
			Группа -- множество, на котором задана бинарная операция, 
			$"\cdot": G\times G \rightarrow G, g_{1}\cdot g_{2} = g_{3}$
			удовлетворяющая следующим свойствам:
		\end{defin}
		
		\begin{enumerate}
			\item Ассоциативность $$g_{1}\cdot\left(g_{2}\cdot g_{3}\right) = \left(g_{1}\cdot g_{2}\right)\cdot g_{3} = g_{1}\cdot g_{2}\cdot g_{3}$$
			\item Существование правой единицы $$\exists \  e: g\cdot e \ \forall g \in G  \ \ g\cdot e = g$$
			\item Существование обратного правого элемента $$\forall g \in G \  \exists \  g^{-1}: \ \  g\cdot g^{-1} = e$$ 
		\end{enumerate}
		
		Как видно из определения, понятие абстрактной группы не подразумевает коммутативности бинарной операции.
		
		\begin{defin}
			Абелева группа -- группа с коммутативной бинарной операцией.
		\end{defin}
		
		Рассмотрим некоторые следствия из определения абстрактной группы:
		
		\begin{cons}
			Обратный элемент к единичному: $e^{-1} = e$.
		\end{cons}		
		
			Действительно, домножим обе части равенства на единичный элемент 
			слева:
			
			\begin{align}
				e^{-1} = e \left.\right| \cdot e \\			
				e\cdot e^{-1} = e\cdot e	
			\end{align}
			
			По определению правой единицы, правая часть равна $e$. В то же 
			время, по определению правого обратного элемента левая часть равна 
			$e$, что приводит к верному равенству.
			
		\begin{cons}
			Обратный элемент к обратному элементу элемента равен самому этому
			элементу.
		\end{cons}
		
			Рассмотрим выражение $\left(g^{-1}\right)^{-1} = g$. 
			Домножим справа на обратный элемент $g^{-1}$.
			
			\begin{align}
				\left(g^{-1}\right)^{-1} = g \left.\right| \cdot g^{-1} \\
				\left(g^{-1}\right)^{-1}\cdot g^{-1} = g\cdot g^{-1}\\
				\left(g^{-1}\right)^{-1}\cdot g^{-1} = e
			\end{align}
			
			А так как по определению $g \cdot g^{-1} = e$, то следствие 
			оказывается верным.
			
		\begin{cons}
			Обратный элемент к произведению элементов $g_{1}$ и $g_{2}$ есть 
			произведение обратных элементов $g_{2}^{-1}$ и $g_{1}^{-1}$.
		\end{cons}
		
			Домножим выражение $\left(g_{1}\cdot g_{2}\right)^{-1} = 
			g_{2}^{-1}\cdot g_{1}^{-1}$ слева на произведение $g_{1}g_{2}$:
			
			
			\begin{align}
				\left(g_{1}\cdot g_{2}\right)\cdot 
				\left(g_{1}\cdot g_{2}\right)^{-1} = 
				\left(g_{1}\cdot g_{2}\right) \cdot g_{2}^{-1}g_{1}^{-1}
			\end{align}
			
			Воспользуемся свойством ассоциативности группового умножения и 
			определением обратного правого элемента:
			
			\begin{align}
				\left(g_{1}\cdot g_{2}\right)\cdot 
				\left(g_{1}\cdot g_{2}\right)^{-1} = 
				g_{1}\cdot \left(g_{2} \cdot g_{2}^{-1}\right)\cdot g_{1}^{-1}
				\\
				\left(g_{1}\cdot g_{2}\right)\cdot 
				\left(g_{1}\cdot g_{2}\right)^{-1} = 
				g_{1}\cdot e \cdot g_{1}^{-1}
			\end{align}
			
			Воспользовавшись еще раз определением правого обратного и правого
			единичного элемента, получаем:
			
			\begin{align}
				\left(g_{1}\cdot g_{2}\right)\cdot 
				\left(g_{1}\cdot g_{2}\right)^{-1} = 
				g_{1}\cdot g_{1}^{-1}\\
				\left(g_{1}\cdot g_{2}\right)\cdot 
				\left(g_{1}\cdot g_{2}\right)^{-1} = 
				e
			\end{align}			 
			
			Получили верное равенство, значит свойство верно.				 
		Дадим еще несколько определений:
		
		\begin{defin}		
			Число элементов группы называется порядком группы.
		\end{defin}
		
		Если порядок группы конечен, то такая группа называется конечной.
		В противном случае группа называется бесконечной.
			
		Приведем некоторые примеры групп:
		
		\begin{exmpl}
			$\left(\mathbb{Z}, +\right)$ -- группа целых чисел по сложению. 
			Свойства следуют из алгебраических свойств множества целых чисел.
			Группа является абелевой.
		\end{exmpl}
		
		\begin{exmpl}
			$\left(\mathbb{Z}/\lbrace 0\rbrace, \cdot \right)$ -- группа целых 
			чисел без нуля по умножению. Свойства аналогично следуют из 
			алгебраических свойств целых чисел. Группа является абелевой.
		\end{exmpl}
		
		\begin{exmpl}
			$\left(\mathbb{Q}, +\right)$ -- группа рациональных чисел по 
			сложению. Нейтральным элементом является 0, обратным является
			число противоположное данному. Группа является абелевой.
		\end{exmpl}
		
		\begin{exmpl}
			$\left(\mathbb{R}, +\right)$ -- группа действительных чисел по 
			сложению. Нейтральным элементом является 0, обратным является
			число противоположное данному. Группа является абелевой.
		\end{exmpl}
		
		\begin{exmpl}
			$\left(\mathbb{C, +}\right)$ -- группа действительных чисел по 
			сложению. Нейтральным элементом является 0, обратным является
			число противоположное данному. Группа является абелевой.
		\end{exmpl}
		
		\begin{exmpl}
			$\left(Z_{n} = \lbrace z \in \mathbb{C}, z = \sqrt[n]{1}\rbrace, 
			\cdot\right)$ -- группа комплексных корней $n$-ой степени из 1 по
			умножению. Групповая структура следует из алгебраических свойств
			комплексных чисел: $e^{i\cdot\frac{a}{2\pi k}} \cdot 
			e^{i\cdot\frac{b}{2\pi k}} = e^{i\cdot\frac{a + b}{2\pi k}}$.
			Нейтральным элементом является единица.
		\end{exmpl}
		
		\begin{exmpl}
			$ S_{n}$ -- группа перестановок порядка n c операцией композиции.
		\end{exmpl}
		
		\begin{exmpl}
			$V$ -- векторное пространство. Векторное пространство обладает 	
			групповой структурой, следующей из определения векторного 
			пространства.
		\end{exmpl}
		
		\begin{exmpl}
			$GL\left(V\right)$ -- группа обратимых преобразований векторного 	
			пространства с операцией композиции. Групповой единицей является 
			тождественное преобразование, обратным элементом -- обратное к
			данному преобразование векторного пространства, существующее по 
			построению.
		\end{exmpl}

		\subsection{Определение подгруппы}		
		
		\begin{defin}
			Подгруппа -- подмножество исходного множества, $H \subset G$ 
			которое само является группой: $H \subset G$.
		\end{defin}
		
		Из определения видно, что выполняются очевидные свойства:
		
		\begin{enumerate}
			\item Замкнутость по операции:
				$$h_{1}, h_{2} \in H, h_{1}h_{2} \in H$$
			\item Существование обратного: 
				$$h \in H, h^{-1} \in H$$
			\item Наличие групповой единицы:
				$$e \in H$$
		\end{enumerate}

		\textit{Замечание:} в любой неабелевой группе можно выделить абелеву
		 подгруппу.
%------------------------------------------------------------------------------			
		\subsection{Определение сопряженной и нормальной подгруппы}

		 Дадим определение сопряженной подгруппы. Рассмотрим произвольную 
		 подгруппу $H$ для группы $G$. Выберем групповой элемент $g, g \in G,
		  g \notin H$.
		
		\begin{defin}
			 Множество $\lbrace ghg^{-1} | h \in H\rbrace$, обладающее
			 групповой структурой, называется подгруппой сопряженной к
			 подгруппе $H$.
		\end{defin}
		
		Если рассмотренная выше сопряженная подгруппа характеризуется тем, что:
		
		$$\forall g \in G, g\cdot H \cdot g^{-1} \neq H$$
		
		То такая подгруппа называется нормальной или инвариантной. Приведем 
		пример нормальной сопряженной подгруппы:
		
		\begin{exmpl}
			Ранее мы рассматривали группу $GL\left(n\right)$ -- группу 
			обратимых преобразований пространства размерности $n$. В такой 
			группе можно рассмотреть следующую подгруппу:
			
			$$SL\left(n\right) = \lbrace g \left.\right| g \in 
			GL\left(n\right), \det g = 1 \rbrace$$
			
			Такая группа называется группой ортогональных преобразований 
			пространства. Доказательство того факта, что множество подобных 
			преобразований является группой относительно операции композиции 
			довольно тривиально. Замкнутость относительно операции композиции 
			следует из свойств детерминанта матриц, единичный элемент исходной 
			группы удовлетворяет условию на элементы подгруппы (единичная 
			матрица), обратные существуют по определению $GL\left(n\right)$.
			
			Свойство инвариантности (нормальности) данной группы индуцируется 
			свойством детерминанта матриц:
			
			$$\det \left(ghg^{-1}\right) =  
			\det\left(g\right)\det\left(h\right)\det\left(g^{-1}\right) = 
			d \cdot 1 \cdot d^{-1} = 1$$
		\end{exmpl}  
		
		\subsection{Определение центра группы}
%------------------------------------------------------------------------------
		
		Рассмотрим понятие центра: $c_{G} = \lbrace c, | gc = cg \forall g \in G\rbrace$
		
		Свойства: 
		\begin{enumerate}
			\item Центр -- подгруппа
			\item $c_{1}c_{2}g = gc_{1}c_{2}$
		\end{enumerate}
		
		Рассмотрим группу $GL\left(3, \mathbb{R}\right)$. Все матрицы, которую коммутируют в этой группе -- матрицы пропорциональны единичной $g = \lambda I, \lambda \neq 0$
		
		Смежные классы
		
		$G \supset H, g_{1} \sim g_{2}$, если $g_{1} = g_{2}h, h \in H$
		Проверим свойства: 
		\begin{enumerate}
			\item $g\sim g, g = ge$
			\item $g_{1} \sim g_{2}, \leftrightarrow g_{2} \sim g_{1}, g_{1} = g_{2}h, g_{2} = g_{1}h^{-1}$
			\item $g_{1} \sim g_{2}, g_{2} \sim g_{3}, \Rightarrow g_{1}\sim g_{3}, g_{1} = g_{2}h_{1}, g_{2} = g_{3}h_{2}, \Rightarrow g_{1} = g_{3}h_{2}h_{1}$
		\end{enumerate}
		
		Отношение эквивалентности разбивает множество на непересекающиеся классы эквивалентности. 
		
		$G\slash H $ -- фактор пространство $G$ по $H$.
		
		Любой выбранный элемент смежного класса называется его представителем. Обозначается как правило $\left[g\right]$.
		
		%Расслоение -- картинка
		
		Смежные классы для неабелевых групп бывают правые и левые. Если подгруппа, по которой берется отношение эквивалентности -- нормальная подгруппа, то левые и правые смежные классы совпадают.
		
		Если $H$ -- нормальная подгруппа, то $G\slash H$ -- факторгруппа.
		
		
		%картинка о факторгруппе
		
		Рассмотрим свойства:
		
		\begin{enumerate}
			\item $g_{1}H, g_{2}H, g_{1}Hg_{2}H = g_{1}g_{2}g_{2}^{-1}Hg_{2}H = g_{1}g_{2}H$
			\item $\left(gH\right)^{-1} = g^{-1}H$
			\item $e' = eH$
		\end{enumerate}
		
		Рассмотрим $GL^{+}\left(n, \mathbb{R}\right) \supset SL\left(n, \mathbb{R}\right)$.
		
		$g = \sqrt[n]{\det g}I\cdot g\frac{1}{\sqrt[n]{\det g}} I$ 
		Что приводит к факторизации нашей группы множеством $\mathbb{R^{+}}$.
		
	\section{Действие групп на множестве}
	
		Классифицируем действия:
		
		\begin{enumerate}
			\item Сдвиги $g: G\leftarrow G g_{1} \leftarrow g_{2} = gg_{1}$
			\item Преобразование подобия (сопряжения) $g: G \leftarrow G g_{1}\leftarrow g_{2} = gg_{1}g^{-1}$
		\end{enumerate}
		
		Возьмем множество $M$, и множество $T$ -- обратимых преобразований множества $M$ в себя. Действие группы $G$ на множестве $M$ реализовано тогда, когда $\forall g \in G \exists t \in T$ и $g_{1}g_{2} \leftarrow t_{g_{1}g_{2}} = t_{g_{1}}\cdot t_{g_{2}}$, $e \leftarrow t_{e} = i$, $g^{-1} \leftarrow t_{g^{-1}} = t^{-1}_{g}$
		
		Если $x \in M, g \leftarrow t_{g}$, то $t_{g}\left(x\right) = y \equiv gx = y$.
		
		$h \in G, h\leftarrow gh \leftarrow g_{2}g_{1}h$
		
		$h \in G, h \leftarrow ghg^{-1} \leftarrow g_{2}g_{1}hg_{1}^{-1}g_{2}^{-1} = g_{2}g_{1}h\left(g_{2}g_{1}\right)^{-1}$
		
		Рассмотрим понятие орбиты: $M \supset M_{x}, M_{x} = Gx$.
		
		Введем $x \sim y \Leftrightarrow y = gx$. Данное отношение -- отношение эквивалентности. Такое отношение эквивалентности разбивает пространство на непересекающиеся классы эквивалентности -- орбиты.
		
		Однородное пространство.
		
		Пространство $M$ называется однородным, если оно состоит из одной орбиты.
		
		Возьмем точку $x$ из однородного пространства $M_{x}$. Рассмотрим множество $H_{x} \subset G = \lbrace h | hx = x\rbrace$ В таком случае, $H_{x}$ -- подгруппа, которая называется стационарной подгруппой точки $x$. Рассмотрев другую точку $y$ и ее стационарную подгруппу $H_{y}$, получаем, что группа $H_{y}$ является сопряженной для группы $H_{x}$.
		
		Утверждение -- однородное пространство назодится во взаимнооднозначном соответствии с факторпространством $G/H_{x}$
		
		Охарактеризуем факторпространство представителями: $\left[g\right]. g = \left[g\right]h$. Рассмотрим точку $y = gx = \left[g\right]h = \left[g\right]x$   
		
		Рассмотрим в качестве примера группу $SO_{3}$  
		
		Орбита вектора -- $S^2$. В таком случае $\mathbb{R}^{3} =\bigcup\limits_{r > 0} S^{2}_{r} \bigcup {\vec{0}}$
		
		Значит, $S^{2} \approx SO_{3}/SO_{2}$
		
		\section{Гомоморфизм групп}
		
		\textit{Гомоморфизм групп} -- отображение $\Phi : G\leftarrow G'$ со свойством сохранения группового умножения:
		
		$\Phi\left(g_{1}g_{2}\right) = \Phi\left(g_{1}\right)\Phi\left(g_{2}\right)$
		
		Некоторые свойства:
		
		$\Phi\left(e\right) = e$
		
		$\Phi\left(g^{-1}\right) = \left(\Phi\left(g\right)\right)^{-1}$
		
		Для $\forall G$ существует тривиальный гомоморфизм в единичный элемент.
		
		Рассмотрим образ гомоморфизма: $\Im \Phi \subset G'$ 
		
		образ гомоморфизма -- подгруппа в $G'$. 
		
		Ядро гомоморфизма -- $\ker \Phi \subset G, \ker \Phi = \lbrace k | \Phi\left(k\right) = e'\rbrace$
		
		Ядро гомоморфизма -- подгруппа в группе $G$ (к тому же элементарной
		
		Если есть нетривиальный гомоморфизм группы в группу обратимых преобразований пространства, то определено действие группы на пространстве.
		
		Гомоморфизм -- отображение $G\slash\ker \Phi \leftarrow G'$
		
		Пример:
		
		$G \supset H, \pi: G \leftarrow G\slash H$, $H$ -- инвариантная подгруппа.
		
		%общее устройство
		
		\textit{Изоморфизм} -- гомоморфизм $\Phi: G \rightarrow G'$, такой что $\Phi^{-1}: G' \rightarrow G$
\end{document}